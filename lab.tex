% lab.tex — example lab document that uses labstyle.sty
\documentclass[a4paper,11pt]{scrartcl}
\KOMAoptions{
  parskip=half,
  headings=small,
  headinclude=true,
  footinclude=true
}
\areaset{175mm}{270mm}

\usepackage{labstyle}

% ------------------------------------------
% Language support (choose ONE)
% ------------------------------------------

% pdfLaTeX:
\usepackage[ngerman]{babel}

% LuaLaTeX / XeLaTeX:
% \usepackage{polyglossia}
% \setmainlanguage{german}

% ------------------------------------------
% Quotation handling
% ------------------------------------------
\usepackage{csquotes}

% % ------------------------------------------
% % Abstract: no side indent, no paragraph indent
% % ------------------------------------------
% \makeatletter
% \renewenvironment{abstract}{%
%   \small
%   \parindent=0pt%
%   \begin{center}\bfseries\abstractname\end{center}%
%   \ignorespaces
% }{\par}
% \makeatother

% ------------------------------------------
% Custom Units
% ------------------------------------------
\DeclareSIUnit{\rpm}{rpm}
\DeclareSIUnit{\newtonmeter}{\newton\metre}

% ------------------------------------------
% Metadata
% ------------------------------------------
\renewcommand{\labtitle}{Praktikum I}
\renewcommand{\labsubtitle}{Cuboid: Regelung eines EC-Motors}
\renewcommand{\labauthor}{RT1}
\renewcommand{\labdate}{\today}

\begin{document}
\makelabtitle

\begin{abstract}
    Dieses Praktikum behandelt die Regelung des Antriebsmotors des \emph{Cuboid}.
    Die Vorlage ist bewusst schlank: klare Typografie, wenige Abhängigkeiten, konsistente Einheiten (\texttt{siunitx}).
\end{abstract}

Dieses Praktikum behandelt die Regelung des Antriebsmotors des \emph{Cuboid}.
Die Vorlage ist bewusst schlank: klare Typografie, wenige Abhängigkeiten, konsistente Einheiten (\texttt{siunitx}).

\section{Einleitung}
Der vorliegende Antrieb besteht aus einem Maxon EC90 Motor mit Maximalstrom \qty{15}{A} und
Maximaldrehzahl \qty{2000}{\rpm} (\(\approx\qty{33.3}{Hz}\), \(\approx\qty{209.4}{\radian\per\second}\)).
%
Die Schwungmasse (Scheibe) wurde entfernt; im Fokus steht der EC-Motor.

In der Praxis werden Drehzahlen häufig in Umdr./min angegeben;
für Modellierung und Regelung arbeiten wir konsequent mit SI-Einheiten (\si{\radian\per\second}).

\medskip
\noindent
\begin{minipage}[t]{0.42\linewidth}
    \centering
    \includegraphics[width=.95\linewidth]{cuboid_foto_sm.jpg}\\
    \small Abbildung: Cuboid (Foto)
\end{minipage}\hfill
\begin{minipage}[t]{0.56\linewidth}
    \centering
    \includegraphics[width=.98\linewidth]{EC90_datasheet.jpg}\\
    \small Abbildung: Ausschnitt aus dem EC90-Datenblatt
\end{minipage}

\medskip
Häufig werden unterschiedliche Begriffe für ähnliche Regelkreise verwendet; meist ist dasselbe gemeint:
\begin{itemize}
    \item Lageregler \(\equiv\) Winkelregler \(\equiv\) Positionsregler
    \item Drehzahlregler \(\equiv\) Geschwindigkeitsregler
\end{itemize}

\section{Systemidentifikation und Modellbildung}
Wir betrachten den Pfad \(i_{\text{soll}}(t)\rightarrow \omega(t)\).
Ein einfaches rotatives Modell lautet:
\[
    J\,\dot{\omega}(t) + b\,\omega(t) \;=\; k_T\,i(t) - \tau_L(t),
\]
mit Trägheitsmoment \(J\), Reibung \(b\), Drehmomentkonstante \(k_T\) und Störmoment \(\tau_L\).

\begin{gentable}{ll}{Relevante Parameter (Beispielwerte)}
    \textbf{Größe} & \textbf{Wert} \\
    \midrule
    Maximalstrom & \(\qty{15}{A}\) \\
    Maximaldrehzahl & \(\qty{2000}{\rpm}\) \\
    Encoderauflösung & \(4\times 6400\) Inkr./Umdrehung \\
\end{gentable}

\begin{aufgaben}
    \item Führen Sie eine Frequenzgangmessung \(i_{\text{soll}}\rightarrow \omega\) durch.
    Hinweis: fast reibungsfrei \(\Rightarrow\) integrierendes Verhalten; einfache P-Regelung ist geeignet.
    \item Modellieren Sie das System aus dem Datenblatt (Newton rotativ) und schätzen Sie \(J,\,b,\,k_T\).
    \item Verbessern Sie das Modell schrittweise (Bandbreite des Stromreglers, Totzeit, Messpfad der Geschwindigkeit).
    \item Skizzieren Sie das gemessene System als Blockschaltbild (mit Einheiten).
\end{aufgaben}

\section{Demonstration aller Stil-Elemente}
\subsection*{Mathematische Kurzformen}
Die Ableitung nach der Zeit wird als
\[
    \frac{\dd \omega}{\dd t}
\]
geschrieben.
Die Lösung eines exponentiellen Ansatzes:
\[
    x(t) = x_0 \,\e^{-\alpha t}.
\]

\subsection*{Automatisches Abbildungsverzeichnis}
\labfig[width=0.7\linewidth]{cuboid_foto_sm.jpg}{Beispiel: automatische Bildumgebung mit Label \texttt{fig:cuboid\_foto\_sm}.}

\subsection*{Aufgaben-Umgebung mit anderem Titel}
\begin{aufgaben}[Zusatzaufgaben]
    \item Diskutieren Sie die Vor- und Nachteile der Modellvereinfachung.
    \item Erweitern Sie Ihr Modell um ein nichtlineares Reibglied.
\end{aufgaben}

\subsection*{Links und Farben}
Weitere Infos unter
\href{https://www.maxongroup.com/}{\color{labColor}Maxon Motor AG}.
Beachten Sie die Farbdefinition \texttt{labColor} aus \texttt{labstyle.sty}.

\section{Drehzahlregelung}
Wir verwenden einen PI-Regler, Ziel: ausreichende Phasenreserve und keine Überschreitung von \qty{15}{A}.

\begin{aufgaben}[Aufgaben — Drehzahlregelung]
    \item Parametrieren Sie \(k_p\) und \(T_n\) für drei Varianten (\qty{500}{ms}, \qty{100}{ms}, \qty{10}{ms}).
    \item Untersuchen Sie Führungsverhalten (Sprung) und Störverhalten (Drehmomentsprung \(\qty{1}{\newtonmeter}\)).
    \item Ergänzen Sie ggf. ein Lead-Glied, um die Phasenreserve zu verbessern.
\end{aufgaben}

\section*{Hinweise}
\begin{itemize}
    \item Bilder liegen in \texttt{figures/} (alternativ \texttt{Figures/}); Pfad ist im Stilpaket gesetzt.
    \item Für Einheiten und Werte stets \verb|\qty| und \verb|\si| aus \texttt{siunitx} verwenden.
    \item Kompilieren Sie mit \texttt{lualatex} oder \texttt{pdflatex}.
\end{itemize}

\begin{answerbox}[3]\end{answerbox}

\begin{answerbox}[6]\end{answerbox}

\begin{answerboxruled}[5]\end{answerboxruled}

\begin{answerboxruled}[8]\end{answerboxruled}

\end{document}
