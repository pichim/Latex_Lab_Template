% lab.tex — example lab document that uses labstyle.sty
\documentclass[a4paper,11pt]{scrartcl}
\KOMAoptions{
  parskip=half,
  headings=small,
  headinclude=true,
  footinclude=true
}
\areaset{175mm}{270mm}

\usepackage{labstyle}

% ------------------------------------------
% Language support (choose ONE)
% ------------------------------------------

% pdfLaTeX:
\usepackage[ngerman]{babel}

% LuaLaTeX / XeLaTeX:
% \usepackage{polyglossia}
% \setmainlanguage{german}

% ------------------------------------------
% Quotation handling
% ------------------------------------------
\usepackage{csquotes}

% % ------------------------------------------
% % Abstract: no side indent, no paragraph indent
% % ------------------------------------------
% \makeatletter
% \renewenvironment{abstract}{%
%   \small
%   \parindent=0pt%
%   \begin{center}\bfseries\abstractname\end{center}%
%   \ignorespaces
% }{\par}
% \makeatother

% ------------------------------------------
% Custom Units
% ------------------------------------------
\DeclareSIUnit{\rpm}{rpm}
\DeclareSIUnit{\newtonmeter}{\newton\metre}

% ------------------------------------------
% Metadata
% ------------------------------------------
\renewcommand{\labtitle}{Praktikum V}
\renewcommand{\labsubtitle}{Drehpendel: Reglerentwurf hängendes und stehendes Pendel mit Hilfe der Wurzelortskurve}
\renewcommand{\labauthor}{Gruppe für Regelungstechnik \& Advanced Control}
\renewcommand{\labdate}{\today}

\begin{document}
\makelabtitle

Aufbauend auf der bestehenden Geschwindigkeitsregelung des DC-Motors wird in diesem Praktikum das System mit angebrachtem Pendel untersucht und geregelt. Es wird vorausgesetzt, dass eine funktionierende Strom- und überlagerte Drehzahlregelkaskade implementiert ist. Der Systemeingang ist somit die Solldrehzahl der Drehzahlregelung. Der Systemausgang ist der Winkel des Pendels. Ziel ist es nun, eine Winkelregelung für das hängende und das stehende Pendel zu entwerfen und zu implementieren. Für den Entwurf der Regler verwenden wir das Wurzelortskurvenverfahren (engl. \emph{Root Locus}).

\textbf{Lernziele}
\begin{itemize}
    \item Linearisierung eines nichtlinearen mechanischen Systems um gegebene Arbeitspunkte
    \item Herleitung und Interpretation von Übertragungsfunktionen
    \item Entwurf eines PI-Reglers mithilfe der Wurzelortskurve (Root Locus)
    \item Umsetzung und Test der Regelung am realen System inkl.\ Enable/Reset-Logik
\end{itemize}

\section{Differentialgleichung}

\labfig[width=0.9\linewidth]{rot_pen_paper_snapshot.png}{Skizze Drehpendel aus \cite{Cazzolato2011Furuta}.}

Die vollständige Herleitung der nichtlinearen Bewegungsgleichungen des Drehpendels ist in \cite{Cazzolato2011Furuta} im Detail beschrieben (2 mechanische DOF $\rightarrow$ 2 nichtlineare Differentialgleichungen 2. Ordnung). Unter Vernachlässigung von Corioliskräften lautet die nichtlineare Differentialgleichung für das Pendel:

% \begin{equation}
\[
    \left(J_p + m_2 l_2^2\right)\ddot{\theta}_2
    + b_2\dot{\theta}_2
    + m_2 g l_2 \sin\theta_2
    = -m_2 L_1 l_2 \cos\theta_2\,\ddot{\theta}_1
\]
% \end{equation}

Um konsistent mit den vergangenen Praktika zu sein, verwenden wir weiterführend für die Winkel anstelle von $\theta_1$ und $\theta_2$ die Bezeichnung $\varphi_1$ für den Winkel des Motors und $\varphi_2$ für den Winkel des Pendels.

\[
    \left(J_p + m_2 l_2^2\right)\ddot{\varphi}_2
    + b_2\dot{\varphi}_2
    + m_2 g l_2 \sin\varphi_2
    = -m_2 L_1 l_2 \cos\varphi_2\,\ddot{\varphi}_1
\]

\textbf{Hierbei ist zu beachten, dass:}

\begin{itemize}
    \item Der Nullwinkel des Pendels ist gegeben, wenn das Pendel hängend ist.
    \item Die Drehrichtung der rotativen Achsen ist mathematisch positiv gewählt (Rechte-Hand-Regel).
\end{itemize}

\begin{gentable}{ll}{Parameter des Drehpendels}
    \textbf{Grösse} & \textbf{Wert} \\
    \midrule
    Armlänge des Motors $L_1$ & $\qty{0.0855}{\metre}$ \\
    Massenträgheitsmoment des Pendels bez. COG $J_p$ & $\qty{6.063e-5}{\kilogram\metre\squared}$ \\
    Masse des Pendels $m_2$ & $\qty{0.028}{\kilogram}$ \\
    Abstand zum Schwerpunkt $l_2$ & $\qty{0.0805}{\metre}$ \\
    Reibungskoeffizient $b_2$ & $\qty{4.4e-5}{\newton\metre\per\radian\per\second}$\\
    Erdbeschleunigung $g$ & $\qty{9.81}{\metre\per\second\squared}$ \\
\end{gentable}

\begin{aufgaben}
    \item Skizzieren Sie die angreifenden Kräfte und Momente am Pendel.
    \item Welche Grösse ist der Systemeingang und welche der Systemausgang?
\end{aufgaben}

\begin{answerboxfillpage}\end{answerboxfillpage}

\section{Pendel hängend}

Im ersten Teil wird das Pendel in seiner grenzstabilen Gleichgewichtslage (wir Vernachlässigen die Reibung), also \emph{hängend}, untersucht. Ziel ist es, das Systemverhalten um diesen Arbeitspunkt zu analysieren und einen geeigneten Regler entwerfen zu können. Dazu wird die nichtlineare Differentialgleichung zunächst linearisiert und in eine Übertragungsfunktion überführt. Anschliessend erfolgt die Bewertung der Stabilität und des statischen Verhaltens, bevor im nächsten Schritt, auf Basis des Wurzelortskurvenverfahrens, der Regler ausgelegt wird.

\subsection*{Systemanalyse}

\begin{aufgaben}
    \item Linearisieren Sie die Differentialgleichung um den Arbeitspunkt des hängenden Pendels ($\varphi_{2,0} = 0$) symbolisch. Nutzen Sie hierfür die Approximation für kleine Winkel: $\sin\varphi_2 \approx \varphi_2$ und $\cos\varphi_2 \approx 1$.
    \item Bestimmen Sie die Übertragungsfunktion von Systemeingang \emph{Winkelbeschleunigung des Motors} $\ddot{\varphi}_1(s)$ zu Systemausgang \emph{Pendelwinkel} $\varphi_2(s)$.
    \item Wir verwenden das Drehpendel mit unterlagerter Drehzahlregelung. Bestimmen Sie deshalb die Übertragungsfunktion von Systemeingang \emph{Winkelgeschwindigkeit des Motors} $\omega_1(s)=\dot{\varphi}_1(s)$ zu Systemausgang \emph{Pendelwinkel} $\varphi_2(s)$. Verändern Sie ausserdem das Vorzeichen der Regelstrecke so, dass ein positiver Eingang zu einer positiven Ausgangsänderung führt. Beachten Sie, dass Sie dieses Vorzeichen unbedingt auch beim realen System berücksichtigen müssen.
    \item Vernachlässigen Sie die Reibung ($b_2 = 0$) und berechnen Sie die Übertragungsfunktion mit Zahlenwerten.
    \item Bestimmen Sie die Pol- und die Nullstellen der Übertragungsfunktion.
    \item Um was für ein System handelt es sich hinsichtlich statischem Verhalten und Stabilität?
\end{aufgaben}

\begin{answerboxfillpage}\end{answerboxfillpage}

\begin{answerbox}[8]\end{answerbox}

\subsection*{Reglerentwurf}

Der zu entwerfende Regler soll das hängende Pendel dämpfen. Mithilfe des Wurzelortskurvenverfahrens wird untersucht, wie sich die Pole des geschlossenen Regelkreises in Abhängigkeit von der Verstärkung in der komplexen Ebene bewegen. Auf dieser Grundlage wird zunächst die Nachstellzeit $T_i$ vorgegeben und anschliessend die erforderliche Verstärkung $k_p$ bestimmt, sodass der geschlossene Regelkreis eine doppelte Polstelle aufweist. Die Dynamik der Drehzahlregelung kann hierbei vernachlässigt werden, da diese deutlich schneller ist als die Dynamik des Pendels.

Als Reglerstruktur wählen wir einen PI-Regler vom Typ:

\[
    C(s) = k_p \left( \frac{T_i s + 1}{T_i s} \right)
\]

\begin{aufgaben}
    \item Bestimmen Sie die Nachstellzeit $T_i$, sodass eine Nullstelle bei $z = -9$ entsteht.
    \item Bestimmen Sie die Übertragungsfunktion des offenen Regelkreises (engl. \emph{Loop}) $L(s)$ mit Zahlenwerten. Kürzen Sie, falls möglich, Pol- und Nullstellen, die nicht in der rechten Halbebene liegen.
    \item Normieren Sie die Übertragungsfunktion des offenen Regelkreises gemäss $k\tilde{L}(s)$.
    \item Skizzieren Sie die Wurzelortskurve und gehen Sie dabei wie folgt vor:
    \begin{enumerate}[label*=\alph*), leftmargin=2em]
        \item Skizzieren Sie die Pol- und Nullstellen von $\tilde{L}(s)$ in der komplexen Ebene.
        \item Bestimmen Sie den Polüberschuss sowie den Schnittpunkt der Asymptoten $s_A$.
        \item Überlegen Sie, welche Teile der reellen Achse zur Wurzelortskurve gehören.
        \item Bestimmen Sie die Kandidaten der Verzweigungspunkte (Abzweig- und Zusammenflusspunkte, engl. \emph{breakaway und break-in points}). Entscheiden Sie, welche der Kandidaten tatsächlich Verzweigungspunkte sind, und tragen Sie diese in die Skizze ein.
        \item Vervollständigen Sie die Skizze der Wurzelortskurve.
        \item Bestimmen Sie anhand des Amplitudenkriteriums (engl. \emph{magnitude condition}) die Verstärkung $k$, sodass eine doppelte reelle Polstelle entsteht. Bestimmen Sie anschliessend die Reglerverstärkung $k_p$.
    \end{enumerate}
\end{aufgaben}

\newpage
\begin{answerboxfillpage}\end{answerboxfillpage}

\newpage
\subsection*{Implementation und Testen der Regelung}

In dem erhaltenen \emph{Simulink}-Template ist die Regelung so implementiert, dass sowohl die PI-Regler selbst als auch der eigentliche Motortreiber nur dann aktiv sind, wenn der Winkel innerhalb eines vorgegebenen Bereichs liegt (siehe \emph{Simulink}-Modell). Grundsätzlich ist der Logikblock so aufgebaut, dass die Regler bei der Aktivierung neu initialisiert werden (Reset der Integratoren). Nichtsdestotrotz ist der Logikblock so implementiert, dass die Regelung sowie die Leistungselektronik nur ein einziges Mal aktiv wird. Gerät das Pendel ausserhalb des Bereichs, so wird die Regelung deaktiviert und der Motor ausgeschaltet. Um das System neu zu initialisieren, muss das gesamte \emph{Simulink}-Modell gestoppt und neu gestartet werden. Überlagern Sie die PI-Winkelregelung der bestehenden Drehzahlregelung inklusive der Enable/Reset-Logik und berücksichtigen Sie das negative Vorzeichen am Solldrehzahl-Eingang der Drehzahlregelung.

\section{Pendel stehend}

Im zweiten Teil des Praktikums wird das Pendel in seiner instabilen Gleichgewichtslage, also \emph{stehend}, untersucht. Wiederum linearisieren wir die nichtlineare Differentialgleichung um diesen Arbeitspunkt und überführen sie in eine Übertragungsfunktion. Äquivalent zum ersten Teil erfolgt die Bewertung der Stabilität und des statischen Verhaltens, bevor im nächsten Schritt, auf Basis des Wurzelortskurvenverfahrens, der stabilisierende Regler ausgelegt wird.

\subsection*{Systemanalyse}

\begin{aufgaben}
    \item Linearisieren Sie die Differentialgleichung um den Arbeitspunkt stehendes Pendel ($\varphi_{2,0} = \pi$) symbolisch. Nutzen Sie hierfür die Approximation für kleine Winkel: $\sin\left(\varphi_2-\pi\right) \approx -\varphi_2$ und $\cos\left(\varphi_2-\pi\right) \approx -1$.
    \item Bestimmen Sie die Übertragungsfunktion von Systemeingang \emph{Winkelgeschwindigkeit des Motors} $\omega_1(s)=\dot{\varphi}_1(s)$ zu Systemausgang \emph{Pendelwinkel} $\varphi_2(s)$. Das Vorzeichen der Regelstrecke muss hier nicht verändert werden. Die Nulllage des Pendels ist um den Arbeitspunkt gegeben, wenn das Pendel steht.
    \item Vernachlässigen Sie die Reibung ($b_2 = 0$) und berechnen Sie die Übertragungsfunktion mit Zahlenwerten.
    \item Bestimmen Sie die Pol- und Nullstellen der Übertragungsfunktion.
    \item Um was für ein System handelt es sich hinsichtlich statischem Verhalten und Stabilität?
\end{aufgaben}

\begin{answerboxfillpage}\end{answerboxfillpage}

\begin{answerbox}[9]\end{answerbox}

\subsection*{Reglerentwurf}

Nun soll der zu entwerfende Regler das stehende Pendel stabilisieren. Ziel ist es also, den instabilen Pol der Regelstrecke in die linke Halbebene zu verschieben. Wie beim hängenden Pendel verwenden wir einen PI-Regler und entwerfen diesen mithilfe des Wurzelortskurvenverfahrens. Die Nullstelle des PI-Reglers wird wiederum vorgegeben, sodass wir uns ausschliesslich auf die Konstruktion der Wurzelortskurve und die Bestimmung der Verstärkung konzentrieren können.

\begin{aufgaben}
    \item Bestimmen Sie die Nachstellzeit $T_i$, sodass eine Nullstelle bei $z = -18$ entsteht.
    \item Bestimmen Sie die Übertragungsfunktion des offenen Regelkreises (engl. \emph{Loop}) $L(s)$ mit Zahlenwerten. Kürzen Sie, falls möglich, Pol- und Nullstellen, die nicht in der rechten Halbebene liegen.
    \item Normieren Sie die Übertragungsfunktion des offenen Regelkreises gemäss $k\tilde{L}(s)$.
    \item Skizzieren Sie die Wurzelortskurve und gehen Sie dabei wie folgt vor:
    \begin{enumerate}[label*=\alph*), leftmargin=2em]
        \item Skizzieren Sie die Pol- und Nullstellen von $\tilde{L}(s)$ in der komplexen Ebene.
        \item Bestimmen Sie den Polüberschuss sowie den Schnittpunkt der Asymptoten $s_A$.
        \item Überlegen Sie, welche Teile der reellen Achse zur Wurzelortskurve gehören.
        \item Bestimmen Sie die Kandidaten der Verzweigungspunkte (Abzweig- und Zusammenflusspunkte, engl. \emph{breakaway} und \emph{break-in points}). Entscheiden Sie, welche der Kandidaten tatsächlich Verzweigungspunkte sind, und tragen Sie diese in die Skizze ein.
        \item Vervollständigen Sie die Skizze der Wurzelortskurve.
        \item Bestimmen Sie anhand des Amplitudenkriteriums (engl. \emph{magnitude condition}) die Verstärkung $k$. Anhand der Wurzelortskurve ist ersichtlich, dass wir zwei Möglichkeiten zur Parametrierung einer doppelten Polstelle des geschlossenen Regelkreises haben. Bestimmen Sie beide Verstärkungen und anschliessend die zugehörigen Reglerverstärkungen $k_p$.
        \item Überlegen Sie sich nun, wie die Verstärkung $k$ gewählt werden kann, sodass wir ein System auslegen, bei dem ein Pol am höchsten Punkt der Wurzelortskurve und ein Pol am tiefsten Punkt der Wurzelortskurve zu liegen kommt. Bestimmen Sie anschliessend die zugehörige Reglerverstärkung $k_p$ und implementieren Sie diesen Regler.
    \end{enumerate}
\end{aufgaben}

\newpage
\begin{answerboxfillpage}\end{answerboxfillpage}

\newpage
\subsection*{Implementation und Testen der Regelung}

Implementieren Sie die Winkelregelung für das stehende Pendel analog zum hängenden Pendel. Beachten Sie, dass das Vorzeichen am Solldrehzahl-Eingang der Drehzahlregelung hier positiv sein muss. Ausserdem ist die Nulllage des Pendels um den Arbeitspunkt gegeben, wenn das Pendel steht. Folglich müssen Sie vom gemessenen Winkel $\varphi_2$ noch $\pi$ abziehen, um den betrachteten Arbeitspunkt zu erhalten.

% -----------------------------------------------------------------------------

\newpage

\section{Einleitung}
Der vorliegende Antrieb besteht aus einem Maxon EC90 Motor mit Maximalstrom \qty{15}{A} und
Maximaldrehzahl \qty{2000}{\rpm} (\(\approx\qty{33.3}{Hz}\), \(\approx\qty{209.4}{\radian\per\second}\)).
%
Die Schwungmasse (Scheibe) wurde entfernt; im Fokus steht der EC-Motor.

In der Praxis werden Drehzahlen häufig in Umdr./min angegeben;
für Modellierung und Regelung arbeiten wir konsequent mit SI-Einheiten (\si{\radian\per\second}).

\medskip
\noindent
\begin{minipage}[t]{0.42\linewidth}
    \centering
    \includegraphics[width=.95\linewidth]{cuboid_foto_sm.jpg}\\
    \small Abbildung: Cuboid (Foto)
\end{minipage}\hfill
\begin{minipage}[t]{0.56\linewidth}
    \centering
    \includegraphics[width=.98\linewidth]{EC90_datasheet.jpg}\\
    \small Abbildung: Ausschnitt aus dem EC90-Datenblatt
\end{minipage}

\medskip
Häufig werden unterschiedliche Begriffe für ähnliche Regelkreise verwendet; meist ist dasselbe gemeint:
\begin{itemize}
    \item Lageregler \(\equiv\) Winkelregler \(\equiv\) Positionsregler
    \item Drehzahlregler \(\equiv\) Geschwindigkeitsregler
\end{itemize}

\section{Systemidentifikation und Modellbildung}
Wir betrachten den Pfad \(i_{\text{soll}}(t)\rightarrow \omega(t)\).
Ein einfaches rotatives Modell lautet:
\[
    J\,\dot{\omega}(t) + b\,\omega(t) \;=\; k_T\,i(t) - \tau_L(t),
\]
mit Trägheitsmoment \(J\), Reibung \(b\), Drehmomentkonstante \(k_T\) und Störmoment \(\tau_L\).

\begin{gentable}{ll}{Relevante Parameter (Beispielwerte)}
    \textbf{Größe} & \textbf{Wert} \\
    \midrule
    Maximalstrom & \(\qty{15}{A}\) \\
    Maximaldrehzahl & \(\qty{2000}{\rpm}\) \\
    Encoderauflösung & \(4\times 6400\) Inkr./Umdrehung \\
\end{gentable}

\begin{aufgaben}
    \item Führen Sie eine Frequenzgangmessung \(i_{\text{soll}}\rightarrow \omega\) durch.
    Hinweis: fast reibungsfrei \(\Rightarrow\) integrierendes Verhalten; einfache P-Regelung ist geeignet.
    \item Modellieren Sie das System aus dem Datenblatt (Newton rotativ) und schätzen Sie \(J,\,b,\,k_T\).
    \item Verbessern Sie das Modell schrittweise (Bandbreite des Stromreglers, Totzeit, Messpfad der Geschwindigkeit).
    \item Skizzieren Sie das gemessene System als Blockschaltbild (mit Einheiten).
\end{aufgaben}

\section{Demonstration aller Stil-Elemente}
\subsection*{Mathematische Kurzformen}
Die Ableitung nach der Zeit wird als
\[
    \frac{\dd \omega}{\dd t}
\]
geschrieben.
Die Lösung eines exponentiellen Ansatzes:
\[
    x(t) = x_0 \,\e^{-\alpha t}.
\]

\subsection*{Automatisches Abbildungsverzeichnis}
\labfig[width=0.7\linewidth]{cuboid_foto_sm.jpg}{Beispiel: automatische Bildumgebung mit Label \texttt{fig:cuboid\_foto\_sm}.}

\subsection*{Aufgaben-Umgebung mit anderem Titel}
\begin{aufgaben}[Zusatzaufgaben]
    \item Diskutieren Sie die Vor- und Nachteile der Modellvereinfachung.
    \item Erweitern Sie Ihr Modell um ein nichtlineares Reibglied.
\end{aufgaben}

\subsection*{Links und Farben}
Weitere Infos unter
\href{https://www.maxongroup.com/}{\color{labColor}Maxon Motor AG}.
Beachten Sie die Farbdefinition \texttt{labColor} aus \texttt{labstyle.sty}.

\section{Drehzahlregelung}
Wir verwenden einen PI-Regler, Ziel: ausreichende Phasenreserve und keine Überschreitung von \qty{15}{A}.

\begin{aufgaben}[Aufgaben — Drehzahlregelung]
    \item Parametrieren Sie \(k_p\) und \(T_n\) für drei Varianten (\qty{500}{ms}, \qty{100}{ms}, \qty{10}{ms}).
    \item Untersuchen Sie Führungsverhalten (Sprung) und Störverhalten (Drehmomentsprung \(\qty{1}{\newtonmeter}\)).
    \item Ergänzen Sie ggf. ein Lead-Glied, um die Phasenreserve zu verbessern.
\end{aufgaben}

\section*{Hinweise}
\begin{itemize}
    \item Bilder liegen in \texttt{figures/} (alternativ \texttt{Figures/}); Pfad ist im Stilpaket gesetzt.
    \item Für Einheiten und Werte stets \verb|\qty| und \verb|\si| aus \texttt{siunitx} verwenden.
    \item Kompilieren Sie mit \texttt{lualatex} oder \texttt{pdflatex}.
\end{itemize}

\begin{answerbox}[3]\end{answerbox}

\begin{answerbox}[6]\end{answerbox}

\begin{answerboxruled}[5]\end{answerboxruled}

\begin{answerboxruled}[8]\end{answerboxruled}

\begin{thebibliography}{1}

    \bibitem{Cazzolato2011Furuta}
    B.~S. Cazzolato and Z.~Prime,
    \newblock ``On the Dynamics of the Furuta Pendulum,''
    \newblock {\em Journal of Control Science and Engineering}, vol.~2011, Article ID~528341, 8~pages.
    \newblock doi: \href{https://doi.org/10.1155/2011/528341}{10.1155/2011/528341}.

\end{thebibliography}

\end{document}
